%% Beginning of file 'sample7.tex'
%%
%% Version 7. Created January 2025.  
%%
%% AASTeX v7 calls the following external packages:
%% times, hyperref, ifthen, hyphens, longtable, xcolor, 
%% bookmarks, array, rotating, ulem, and lineno 
%%
%% RevTeX is no longer used in AASTeX v7.
%%
\documentclass[linenumbers,trackchanges]{aastex7}
%%
%% This initial command takes arguments that can be used to easily modify 
%% the output of the compiled manuscript. Any combination of arguments can be 
%% invoked like this:
%%
%% \documentclass[argument1,argument2,argument3,...]{aastex7}
%%
%% Six of the arguments are typestting options. They are:
%%
%%  twocolumn   : two text columns, 10 point font, single spaced article.
%%                This is the most compact and represent the final published
%%                derived PDF copy of the accepted manuscript from the publisher
%%  default     : one text column, 10 point font, single spaced (default).
%%  manuscript  : one text column, 12 point font, double spaced article.
%%  preprint    : one text column, 12 point font, single spaced article.  
%%  preprint2   : two text columns, 12 point font, single spaced article.
%%  modern      : a stylish, single text column, 12 point font, article with
%% 		  wider left and right margins. This uses the Daniel
%% 		  Foreman-Mackey and David Hogg design.
%%
%% Note that you can submit to the AAS Journals in any of these 6 styles.
%%
%% There are other optional arguments one can invoke to allow other stylistic
%% actions. The available options are:
%%
%%   astrosymb    : Loads Astrosymb font and define \astrocommands. 
%%   tighten      : Makes baselineskip slightly smaller, only works with 
%%                  the twocolumn substyle.
%%   times        : uses times font instead of the default.
%%   linenumbers  : turn on linenumbering. Note this is mandatory for AAS
%%                  Journal submissions and revisions.
%%   trackchanges : Shows added text in bold.
%%   longauthor   : Do not use the more compressed footnote style (default) for 
%%                  the author/collaboration/affiliations. Instead print all
%%                  affiliation information after each name. Creates a much 
%%                  longer author list but may be desirable for short 
%%                  author papers.
%% twocolappendix : make 2 column appendix.
%%   anonymous    : Do not show the authors, affiliations, acknowledgments,
%%                  and author contributions for dual anonymous review.
%%  resetfootnote : Reset footnotes to 1 in the body of the manuscript.
%%                  Useful when there are a lot of authors and affiliations
%%		    in the front matter.
%%   longbib      : Print article titles in the references. This option
%% 		    is mandatory for PSJ manuscripts.
%%
%% Since v6, AASTeX has included \hyperref support. While we have built in 
%% specific %% defaults into the classfile you can manually override them 
%% with the \hypersetup command. For example,
%%
%% \hypersetup{linkcolor=red,citecolor=green,filecolor=cyan,urlcolor=magenta}
%%
%% will change the color of the internal links to red, the links to the
%% bibliography to green, the file links to cyan, and the external links to
%% magenta. Additional information on \hyperref options can be found here:
%% https://www.tug.org/applications/hyperref/manual.html#x1-40003
%%
%% The "bookmarks" has been changed to "true" in hyperref
%% to improve the accessibility of the compiled pdf file.
%%
%% If you want to create your own macros, you can do so
%% using \newcommand. Your macros should appear before
%% the \begin{document} command.
%%
\newcommand{\vdag}{(v)^\dagger}
\newcommand\aastex{AAS\TeX}
\newcommand\latex{La\TeX}
%%%%%%%%%%%%%%%%%%%%%%%%%%%%%%%%%%%%%%%%%%%%%%%%%%%%%%%%%%%%%%%%%%%%%%%%%%%%%%%%
%%
%% The following section outlines numerous optional output that
%% can be displayed in the front matter or as running meta-data.
%%
%% Running header information. A short title on odd pages and 
%% short author list on even pages. Note that this
%% information may be modified in production.
%%\shorttitle{AASTeX v7 Sample article}
%%\shortauthors{The Terra Mater collaboration}
%%
%% Include dates for submitted, revised, and accepted.
%%\received{February 1, 2025}
%%\revised{March 1, 2025}
%%\accepted{\today}
%%
%% Indicate AAS Journal the manuscript was submitted to.
%%\submitjournal{PSJ}
%% Note that this command adds "Submitted to " the argument.
%%
%% You can add a light gray and diagonal water-mark to the first page 
%% with this command:
%% \watermark{text}
%% where "text", e.g. DRAFT, is the text to appear.  If the text is 
%% long you can control the water-mark size with:
%% \setwatermarkfontsize{dimension}
%% where dimension is any recognized LaTeX dimension, e.g. pt, in, etc.
%%%%%%%%%%%%%%%%%%%%%%%%%%%%%%%%%%%%%%%%%%%%%%%%%%%%%%%%%%%%%%%%%%%%%%%%%%%%%%%%
%%
%% Use this command to indicate a subdirectory where figures are located.
%%\graphicspath{{./}{figures/}}
%% This is the end of the preamble.  Indicate the beginning of the
%% manuscript itself with \begin{document}.

\begin{document}

\title{The Evolution of Stellar Disk/Bulge Morphology after a Major Merger}

%% A significant change from AASTeX v6+ is in the author blocks. Now an email
%% address is required for each author. This means that each author requires
%% at least one of the following:
%%
%% \author
%% \affiliation
%% \email
%%
%% If these three commands are not available for each author, the latex
%% compiler will issue an error and if you force the latex compiler to continue,
%% it will generate an incomplete pdf.
%%
%% Multiple \affiliation commands are allowed and authors can also include
%% an optional \altaffiliation to indicate a status, i.e. Hubble Fellow. 
%% while affiliations are indexed as footnotes, altaffiliations are noted with
%% with a non-numeric footnote that is set away from the numeric \affiliation 
%% footnotes. NOTE that if an \altaffiliation command is used it must 
%% come BEFORE the \affiliation call, right after the \author command, in 
%% order to place the footnotes in the proper location. Because non-numeric
%% symbols are used, \altaffiliation should be used sparingly.
%%
%% In v7 the \author command takes an optional argument which provides 
%% additional metadata about the author. Authors can provide the 16 digit 
%% ORCID, the surname (family or last) name, the given (first or fore-) name, 
%% and a name suffix, e.g. "Jr.". The syntax is:
%%
%% \author[orcid=0000-0002-9072-1121,gname=Gregory,sname=Schwarz]{Greg Schwarz}
%%
%% This name metadata in not shown, it is only for parsing by the peer review
%% system so authors can be more easily identified. This name information will
%% also be sent to the publisher so they can include it in the CROSSREF 
%% metadata. Including an orcid will hyperlink the author name to the 
%% author's ORCID page. Note that  during compilation, LaTeX will do some 
%% limited checking of the format of the ID to make sure it is valid. If 
%% the "orcid-ID.png" image file is  present or in the LaTeX pathway, the 
%% ORCID icon will appear next to the authors name.
%%
%% Even though emails are now required for each author, the \email does not
%% produce output in the compiled manuscript unless the optional "show" command
%% is used. For example,
%%
%% \email[show]{greg.schwarz@aas.org}
%%
%% All "shown" emails are show in the bottom left of the first page. Due to
%% space constraints, only a few emails should be shown. 
%%
%% To identify a corresponding author, use the \correspondingauthor command.
%% The command appends "Corresponding Author: " to the argument it appears at
%% the bottom left of the first page like the output from \email. 

\author{Nikhil Garuda}
\affiliation{University of Arizona}
\email{nikhilgaruda@arizona.edu}

%% Use the \collaboration command to identify collaborations. This command
%% takes an optional argument that is either a number or the word "all"
%% which tells the compiler how many of the authors above the command to
%% show. For example "\collaboration[all]{(DELVE Collaboration)}" wil include
%% all the authors above this command.
%%
%% Mark off the abstract in the ``abstract'' environment. 
\begin{abstract}



\end{abstract}

%% Keywords should appear after the \end{abstract} command. 
%% The AAS Journals now uses Unified Astronomy Thesaurus (UAT) concepts:
%% https://astrothesaurus.org
%% You will be asked to selected these concepts during the submission process
%% but this old "keyword" functionality is maintained in case authors want
%% to include these concepts in their preprints.
%%
%% You can use the \uat command to link your UAT concepts back its source.
% \keywords{\uat{Galaxies}{573} --- \uat{Cosmology}{343} --- \uat{High Energy astrophysics}{739} --- \uat{Interstellar medium}{847} --- \uat{Stellar astronomy}{1583} --- \uat{Solar physics}{1476}}

%% From the front matter, we move on to the body of the paper.
%% Sections are demarcated by \section and \subsection, respectively.
%% Observe the use of the LaTeX \label
%% command after the \subsection to give a symbolic KEY to the
%% subsection for cross-referencing in a \ref command.
%% You can use LaTeX's \ref and \label commands to keep track of
%% cross-references to sections, equations, tables, and figures.
%% That way, if you change the order of any elements, LaTeX will
%% automatically renumber them.

\section{Introduction}
For this paper, we aim to investigate how well the remnant of the mergers can be described as a classical elliptical galaxy based on its surface density profile, with a specific focus on determining the best-fit Sérsic profile of the remnant galaxy post-merger. We will only focus on dry mergers, which are mergers between gas-poor galaxies \citep{lin_redshift_2008}. The classification of galaxies based on morphology has been one of the earliest approaches to understanding galaxy structure, particularly based on the relative prominence of their two main components, i.e. the (spheroidal) bulge and the (exponential) disc. This particular scheme was originally proposed by \cite{hubble_extragalactic_1926} and later refined by \cite{flugge_classification_1959}. 

The morphological framework has sparked a fundamental question in the field of galaxy formation and evolution: What is the relationship between a galaxy's shape and its evolutionary history?
By examining this, we can inform our understanding of the role of various astrophysical processes, including mergers, star formation, and black hole growth, in driving galaxy evolution \citep{barnes_dynamics_1992}. 


Our current understanding of galaxy evolution suggests that mergers play a crucial role in shaping the structure and morphology of galaxies. One of the primary ways that mergers can be identified is through their effect on galaxy structure, with merging galaxies often exhibiting peculiar or distorted morphologies (e.g. \cite{toomre_galactic_1972, conselice_symmetry_1997}). Various methods have been developed to quantify the merger fraction, including the CAS approach \citep{conselice_relationship_2003}, the Gini / M20 parameters \citep{lotz_new_2004} and the multi-mode statistics \citep{freeman_new_2013}. These methods can be used to calculate the merger fraction, which is the number of mergers within a given population of galaxies, and to study the role of mergers in shaping galaxy evolution (e.g. \cite{de_propris_millennium_2007}). 

Despite the importance of galaxy mergers in understanding galaxy evolution, there are still many open questions surrounding the galaxy-galaxy merger rate and its consequences. Theoretical and observational studies have yielded inconsistent results, with different groups and simulations producing varying estimates of the merger rate as a function of galaxy mass and merger mass ratio (e.g. \cite{gottlober_merging_2001, weinzirl_bulge_2008}). Furthermore, the role of minor mergers versus major mergers in bulge formation remains a topic of debate, with some studies suggesting that minor mergers can be just as effective as major mergers in building bulge mass \citep{cox_effect_2008}. The merger history of galaxies and how it affects their evolution is also not well understood, with many questions remaining about the typical merger history through which most of the bulge mass in the universe was assembled. 

\section{Proposal}
\subsection{Proposal}
We determine how well is the remnant described as a classical elliptical galaxy based on its surface density profile? What is the best fit sersic profile of these galaxies post merger?

\subsection{Methods}
% - Merger starts at 6.3 (snap 441) - 6.7 (snap 469) Gyr
% - we will be using the data from snap 469 as the it's a few Myr right after the MW-M31 merger. it is more stable and allows the system to relax a bit
% - we will get the particles of MW and M31 for the disk buldge (combine them) from the previous HWs. use the hi-res particles \cite{vandermarelM31VELOCITYVECTOR2012}
% - use the surface mass density code from lab 6
% - use the $I(r) = I_e exp^{-7.67 ( (r/R_e)^{1/n} - 1)}$ to get the sersic profile of the remenant assuming it is elliptical
% - fit for the sersic profile using \texttt{scipy.optimise}

We will utilize the simulation data \citep{vandermarelM31VELOCITYVECTOR2012} from the merger, which occurs between 6.3 (snap 441) and 6.7 (snap 469) Gyr, and focus on the data from snap 469, as it represents a time shortly after the merger (a few Myr) when the system has had a chance to relax and become more stable. We will combine the particles from the disk and bulge of both the MW and M31 using the high-resolution version of particle data (VHighRes), and calculate the surface density profile of the remnant galaxy using the surface mass density code from Lab 6. We will then assume that the remnant galaxy can be described by a Sérsic profile for an elliptical galaxy, given by the equation $I(r) = I_e ~e^{-7.67 ( (r/R_e)^{1/n} - 1)}$, and use the \texttt{scipy.optimize} module to fit the Sérsic profile to the surface density profile, estimating the best-fit parameters, including the effective radius $R_e$, the Sérsic index $n$, and the intensity $I_e$.

We expect to see a surface density profile as shown in Figure \ref{fig:surface-dens-profile} and will fit the Sérsic index to the density profile.

\begin{figure}[h!]
    \centering
    \includegraphics[width=0.6\linewidth]{image.png}
    \caption{Adapted from \cite{vandermarelM31VELOCITYVECTOR2012}, this figure shows the projected surface-density profile (red) of luminous MW and M31 particles in the merger remnant at the end of the N-body simulation.}
    \label{fig:surface-dens-profile}
\end{figure}

\subsection{Hypothesis}
Based on our understanding of galaxy mergers and the formation of elliptical galaxies, we hypothesize that the remnant of the MW-M31 merger would be an elliptical galaxy and will be well-described by a Sérsic profile with a high Sérsic index ($n = 4$) following the de Vaucouleurs profile. Additionally, the simulated remnant galaxy is expected to have  more radially extended than its progenitor galaxies. We predict that the effective radius ($R_e$) of the remnant galaxy will be larger than that of the individual progenitor galaxies, due to the increased size and scale of the merged system. Overall, our hypothesis is that the remnant galaxy will exhibit a density profile that is consistent with that of a classical elliptical galaxy, with a high Sérsic index, a large effective radius, and a surface-density profile that follows a de Vaucouleurs $R^{1/4}$ law.


%% For this sample we use BibTeX plus aasjournalv7.bst to generate the
%% the bibliography. The sample7.bib file was populated from ADS. To
%% get the citations to show in the compiled file do the following:
%%
%% pdflatex sample7.tex
%% bibtext sample7
%% pdflatex sample7.tex
%% pdflatex sample7.tex

\bibliography{sample7}{}
\bibliographystyle{aasjournalv7}

%% This command is needed to show the entire author+affiliation list when
%% the collaboration and author truncation commands are used.  It has to
%% go at the end of the manuscript.
%\allauthors

%% Include this line if you are using the \added, \replaced, \deleted
%% commands to see a summary list of all changes at the end of the article.
%\listofchanges

\end{document}

% End of file `sample7.tex'.
